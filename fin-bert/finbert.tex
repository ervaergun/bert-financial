\documentclass[11pt]{article}
\usepackage[utf8]{inputenc}
\usepackage{url}
\usepackage{chapterbib}
\usepackage[sectionbib]{natbib}
\usepackage[utf8]{inputenc}
\usepackage[T1]{fontenc}


%\title{\vspace{-4cm} Predicting Financial Sentiment Through Pretrained
%  Language Models}
\title{\vspace{-4cm} FinSentiment: Predicting Financial Sentiment Through Transfer Learning}
\author{}
\date{}

\begin{document}

\maketitle

% CANDIDATE JOURNALS
% Quantitative Finance 
% Rview of Financial Studies
% Review of Finance
% Mathematical Finance
%Journal of Financial and Quantitative Analysis
%Finance Research Letters
%Review of Quantitative Finance and Accounting

\section{Abstract}

% somewhat checked
Financial sector accumulates large amount of financial communication
text. There is an increasing interest in the financial text mining tasks. Over the past few years, Natural Language Processing (NLP) based on deep learning
has advanced significantly and rapidly. Significant progress has been made with deep
learning, which shows promising results on financial text mining
models. However, financial sentiment analysis is still a demanding task due to its specialized
language and lack of labeled data in financial domain. General purpose
machine learning methods are not as effective due to specialized
language used in the financial context. To address this issue, we use pre-trained language models to analyze sentiment
in financial text since these models can be further trained on domain specific
corpora and they require fewer labeled examples to tackle NLP tasks in financial domain.
We introduce FinSentiment, which is semantic

We introduce financial variants of various pre-trained language models
based on BERT and GPT2 trained on large-scale corpora to tackle NLP sentiment analysis tasks in
financial domain. We also propose variants of models simultaneously
trained on general corpora and financial domain corpora. Our results show that our models outperforms all current
state-of-the-art model for two financial sentiment analysis datasets. We find that the proposed modified methods outperforms state-of-the-art machine learning
methods even with a smaller training set and fine-tuning only a part of
the models. Extensive experimental results demonstrate the effectiveness
and robustness of especially xxx. The datasets, source-code, and
pre-trained models are available online.

\section{Introduction}

% not checked
Financial text data are utilized to predict and analyze future market trends in finance and economics~\cite{xxx}. Financial text mining plays an important role in financial technology,
whether for official company announcements or analyst reports. The volume of financial text data continues to keep increasing since an unprecedented number of such texts are created daily.
As a result, manual analysis of such texts and gaining actionable insights from them as a result of this analysis is a quite difficult task.
Recent progress in machine learning have made financial text mining models in FinTech possible. Nevertheless, supervised training data construction is prohibitively expensive since it requires the
use of expert knowledge in financial domain. Due to the small amount of labeled training data that can be used for
financial text mining tasks, most financial text mining models cannot directly utilize recently-developed deep learning
techniques~\cite{xxx}.

In this paper, we focus on polarity analysis, which is classifying text as positive, negative or neutral,
in a specific domain. It requires to address two challenges: 1) The
most sophisticated classification methods that make use of neural
nets require vast amounts of labeled data and labeling financial
text snippets requires costly expertise. 2) The sentiment analysis
models trained on general corpora are not suited to the task, because
financial texts have a specialized language with unique vocabulary
and have a tendency to use vague.

% checked
One solution is to use thoroughly and manually crafted financial sentiment lexicons since they incorporate existing financial semantics into textual analysis~\cite{loughran2016}.
However, such methods are based on word counting, which 
fail to analyze deeper semantic meaning of provided text.

Language models have become a key step to achieve state-of-the art results in many different Natural Language Processing (NLP)
tasks. Leveraging the huge amount of unlabeled texts nowadays available, they provide an efficient way to pre-train continuous word
representations that can be fine-tuned for a downstream task, along with their contextualization at the sentence level. This has been
widely demonstrated for English using contextualized
representations~\cite{dai2015, peters2018, howard2018, radford2018,
  devlin2019, yang2019}.

NLP transfer learning methods look like a promising solution
to both of the challenges mentioned above. The core idea behind these models is that by training
language models on very large corpora and then initializing down-stream models with the weights learned from the language
modeling task, a much better performance can be achieved. The
initialized layers can range from the single word embedding layer~\cite{peters2018} to the whole model~\cite{howard2018}. This approach should, in theory, be an
answer to the scarcity of labeled data problem. Language models
don’t require any labels, since the task is predicting the next word.
They can learn how to represent the semantic information. That
leaves the fine-tuning on labeled data only the task of learning how
to use this semantic information to predict the labels.

One particular component of the transfer learning methods is the
ability to further pre-train the language models on domain specific
unlabeled corpus. Thus, the model can learn the semantic relations
in the text of the target domain, which is likely to have a different
distribution than a general corpus. This approach is especially
promising for a niche domain like finance, since the language and
vocabulary used is dramatically different than a general one.

As a summary, we have the following main contributions in this paper:

\begin{itemize}
\item We introduce financial extensions of various BERT and GPT models
  for financial NLP tasks to transfer knowledge from financial domain
  corpora. We evaluate the proposed methods on two
   financial sentiment analysis datasets.
\item We conduct number of experiments on several Financial PhraseBank and FiQA sentiment scoring
  benchmark datasets. We achieve the state-of-the-art results on both financial datasets.
\item We compare the performance of various BERT and GPT models and
  understand why certain models can better explain financial text
  mining tasks. Our models are capable of efficiently capturing language knowledge and semantic
information in large-scale pre-training corpora.
\item We have investigated several aspects of the introduced models, including: effects of further pre-training on financial
corpus, training strategies to prevent catastrophic forgetting and fine-tuning only a small subset of model layers
for decreasing training time without a significant drop in performance.
  \item We implemented our BERT algorithms on Tensorflow and Hugginface
  frameworks. We make the source code and pre-trained models publicly
  available. With minimal task-specific architecture modifications.
\end{itemize}

% not gone over
Catastrophic forgetting~\cite{mccloskey1989} is a common problem in
transfer learning, which means the pre-trained knowledge can be removed during learning of new knowledge.
Therefore, we also investigate whether BERT suffers from the catastrophic forgetting problem.



\subsection{Related Work}

There are previous attempts to use BERT models on financial
corpora~\cite{araci2019, liu2020, yang2020}.

Another work~\cite{yang2018}. has employed high level semantic representations and methods
of inductive transfer learning for NLP by using ULMFit~\cite{howard2018}.

There is other work on detecting semantic orientations~\cite{malo2014}.

%GPT2 Paper
Natural language processing tasks, such as question answering, machine translation, reading comprehension, and summarization, are typically
approached with supervised learning on task specific datasets. We demonstrate that language
models begin to learn these tasks without any explicit supervision when trained on a new dataset
of millions of webpages called WebText. When conditioned on a document plus questions, the answers generated by the language model reach 55
F1 on the CoQA dataset - matching or exceeding the performance of $3$ out of $4$ baseline systems
without using the 127,000+ training examples. The capacity of the language model is essential
to the success of zero-shot task transfer and increasing it improves performance in a log-linear
fashion across tasks. Our largest model, GPT-2, is a 1.5B parameter Transformer that achieves
state of the art results on 7 out of 8 tested language modeling datasets in a zero-shot setting
but still underfits WebText. Samples from the model reflect these improvements and contain coherent paragraphs of text. These findings suggest
a promising path towards building language processing systems which learn to perform tasks from
their naturally occurring demonstrations~\cite{radford2019} 

The use of robo-readers to analyze news texts is an emerging technology trend in computational finance. In recent
research, a substantial effort has been invested to develop sophisticated financial polarity-lexicons that can be used to
investigate how financial sentiments relate to future company performance. However, based on experience from other
fields, where sentiment analysis is commonly applied, it is well-known
that the overall semantic orientation of a sentence may differ from
the prior polarity of individual words.


\section{Experiments}

Using BERT models in finance have also been used in the
literature~\cite{araci2019, yang2020}. Other pretrained models have
also been used~\cite{yang2018}.
%FinBERT: A Pretrained Language Model for Financial Communications.

We do not have many financial datasets for financial tasks.

%Value Investor's club as discussed in~\cite{yang2018}.

\subsection{FiQA}

The provided training dataset for WWW ’18~\cite{maia2018} contains a
total of $1174$ examples from news headlines and tweets. Each example
contains the sentence and the sentence snippet associated with the target entity, aspect, and sentiment score. A
sample FiQA entry is shown in~\ref{tab:fiqa_example}. A Level 1 Aspect
label takes on one of $4$ possible labels (Corporate, Economy, Market or Stock), and our Level 2 Aspect label takes on
one of $27$ possible labels (Appointment, Risks, Dividend Policy, Financial, Legal, Volatility, Coverage, Price Action, etc.). The original
dataset contained a small number of multilabel examples, however, we considered this number
too few to train a meaningful multilabel classifier. Thus, we slightly
stray from the original WWW ’18 task for the purpose of this research. Finally, sentiment score takes on a continuous value between $-1$ and $1$ – most negative to most positive.

We can use FiQA dataset for both classification purpose in terms of aspect levels, and regression purpose in terms of sentiment score.

\begin{table}
  \begin{tabular}{|c|c|}
    \hline 
 Sentence & easyJet expects resilient demand to withstand security
            fears. \\ \hline
Aspect Level 1 &  Corporate \\ \hline 
Aspect Level 2 & Risks \\ \hline 
Sentiment Score & 0.165 \\ \hline 
Target & easyJet \\ \hline 
\end{tabular}
\caption{An example entry from FiQA}
\label{tab:fiqa_example}
\end{table}

A large collection of financial reports published annually by publicly-traded companies is employed to conduct our experiments;
moreover, two analytical techniques – regression and ranking methods –
are applied to conduct these analyses~\cite{xxx}.

Reuter's and Bloomberg dataset: Datasets are in Emre's email.
\url{https://github.com/philipperemy/financial-news-dataset}

Reuter's news dataset:
\url{https://github.com/duynht/financial-news-dataset}

\textbf{NLTK's corpus}
\url{https://www.kaggle.com/boldy717/reutersnltk#__sid=js0}

\textbf{Fed Meeting Notes}
\url{https://fraser.stlouisfed.org/title/federal-open-market-committee-meeting-minutes-transcripts-documents-677?browse=2020s}

FOMC Statements Scraper
https://github.com/souljourner/FOMC-Statements-Minutes-Scraper
Some cleaned transcripts
https://github.com/ali-wetrill/FOMCTranscriptAnalysis

Another source for datasets: \url{https://rstudio-pubs-static.s3.amazonaws.com/495650_c9c874694f164fb5948031801079157f.html#3_data}

\url{https://sraf.nd.edu/textual-analysis/resources/#LM%20Sentiment%20Word%20Lists}

Finally, the texts are stemmed using the Porter stemmer when needed.

10K dataset together with volatilities
http://ifs.tuwien.ac.at/~admire/financialvolatility/


% well-written
Recently, unsupervised pre-training of language models on large corpora has significantly improved the performance of many NLP tasks. The
language models are pretained on generic corpora such as Wikipedia. However, sentiment analysis
is a strongly domain dependent task. Financial sector has accumulated large scale of text of financial and business communications. Therefore,
leveraging the success of unsupervised pretraining and large amount of
financial text could potentially benefit wide range of financial applications.

Recent progress in pre-trained neural language models has significantly improved the performance of many natural language processing (NLP) tasks. In this paper we
propose a new model architecture DeBERTa (Decoding-enhanced BERT with disentangled attention) that improves the BERT and RoBERTa models using two novel
techniques. The first is the disentangled attention mechanism, where each word is represented using two vectors that encode its content and position, respectively,
and the attention weights among words are computed using disentangled
matrices on their contents and relative positions. Second, an enhanced mask decoder is used
to replace the output softmax layer to predict the masked tokens for model pretraining. We show that these two techniques significantly improve the efficiency of
model pre-training and performance of downstream tasks. Compared to RoBERTaLarge, a DeBERTa model trained on half of the training data performs consistently
better on a wide range of NLP tasks, achieving improvements on MNLI.

Recent progress in pre-trained neural language models has significantly improved the performance of many natural language processing (NLP) tasks.

We show that these two techniques significantly improve the efficiency of
model pre-training and performance of downstream tasks. In financial
domain, the same is observed.

BERT and its variants have significantly enhanced word vector representation. Here, we will focus on specific BERT application on financial
datasets.


\subsection{Financial Phrasebank}


http://www.cs.cmu.edu/~ark/10K/data/
Metadata var

10K downloads: https://pypi.org/project/sec-edgar-downloader/. Filing
date is in each text file.

extract MDA and tokenize new files in Noah's website can be used to clean the dataset.
CIK Ticker Mapping: https://www.sec.gov/include/ticker.txt



\subsection{Corporate Reports 10-K \& 10-Q} The most important text data in finance and business communication is corporate report. In the United States,
the Securities Exchange Commission (SEC) mandates all publicly traded companies to file annual
reports, known as Form 10-K, and quarterly reports, known as Form 10-Q. This document provides a comprehensive overview of the company’s
business and financial condition. Laws and regulations prohibit companies from making materially
false or misleading statements in the 10-Ks. The
Form 10-Ks and 10-Qs are publicly available and can be accesses from
SEC website. We obtain 60,490 Form 10-Ks and 142,622
Form 10-Qs of Russell 3000 firms during 1994 and
2019 from SEC website. We only include sections that are textual components, such as Item 1 (Business) in 10-Ks, Item 1A (Risk Factors) in both 10-
Ks and 10-Qs and Item 7 (Managements Discussion and Analysis) in 10-Ks.
\section{Method}
\subsection{Pre-process}
Before feeding 10-K corpus~\cite{kogan2009} for pre-training level, each sentence is separated line by line and  an empty line is inserted between each different document.
\newline
In the sentiment analysis part, FIQA datasets are labelled using the sentiment-scores of each sentence. Negative sentiment-scores are labelled as -1, positive sentiment-scores are labelled as 1 and sentiment-scores equals to 0 are labelled as 0. 
\newline
A handicap of such labelling in fiqa dataset is having far too less amount of 0-labelled sentences. This could be one of the reasons that they are less successful in sentiment analysis of Financial Phrase Databank dataset. We would expect that the model trained with fiqa headline will perform better than other models trained by FPD datasets for predicting fiqa post labels and vice versa.  As expected, FPD datasets reveal better performance than fiqa datasets for FPD datasets. 
\subsection{Transformer-based Algorithms}
First level of the Bert algorithm application is pre-training. In this level, we used transformers from hugging face library~\cite{wolf2019huggingface}. \textit{Transformers} is an open source library allowing users to pre-train and test their models with various Bert-variant algorithms. Its structure has an encoder\& decoder-based network enhanced by attention mechanism utilizing recurrence and convolution \cite{vaswani2017attention}.  
\newline Here, we tested Bert, Roberta and Xlnet algorithms using 10-K corpus~\cite{kogan2009} on pre-training level and data from FIQA \& Financial Phrase Databank on sentiment analysis level.

\bibliographystyle{plain}
\bibliography{finbert}


\end{document}
