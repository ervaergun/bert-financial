\documentclass[11pt]{article}
\usepackage[utf8]{inputenc}
\usepackage{url}
\usepackage{chapterbib}
\usepackage[sectionbib]{natbib}
\usepackage[utf8]{inputenc}
\usepackage[T1]{fontenc}


\title{\vspace{-4cm} Predicting Stock Returns and Risk from Financial Reports with Pretrained Language Models}
\author{}
\date{}

% Candidate Journals
% Quantitative Finance
% Rview of Financial Studies
% Review of Finance
% Mathematical Finance
%Journal of Financial and Quantitative Analysis
%Finance Research Letters
%Review of Quantitative Finance and Accounting


\begin{document}

\maketitle

\section{Abstract}

\section{Introduction and Related Work}

%related work
There are previous methods that use Machine learning to predict stock
markets.~\cite{ding2015} proposes a deep learning method for
event-driven stock market prediction. First, events are extracted from news text, and represented as dense vectors,
trained using a novel neural tensor network. Second, a deep convolutional neural network is used to model both short-term and
long-term influences of events on stock price movements.

The use of robo-readers to analyze news texts is an emerging technology trend in computational finance. In recent
research, a substantial effort has been invested to develop sophisticated financial polarity-lexicons that can be used to
investigate how financial sentiments relate to future company performance. However, based on experience from other
fields, where sentiment analysis is commonly applied, it is well-known
that the overall semantic orientation of a sentence may differ from the prior polarity of individual words.

% Neural network model
Financial risk, defined as the chance to deviate from return expectations, is most commonly measured
with volatility. Due to its value for investment decision making, volatility prediction is probably
among the most important tasks in finance and risk management. Although evidence exists that enriching purely financial models with natural language
information can improve predictions of volatility, this task is still comparably underexplored. We introduce PRoFET, the
first neural model for volatility prediction jointly exploiting both
semantic language representations and a comprehensive set of financial
features. As language data, we use transcripts from quarterly
recurring events, so-called earnings calls; in these calls, the performance of publicly traded companies is summarized and prognosticated by their management. We show that our
proposed architecture, which models verbal context with an attention mechanism, significantly outperforms the previous state-of-the-art and other strong
baselines. Finally, we visualize this attention mechanism on the token-level, thus aiding interpretability and providing a use case of PRoFET as a tool
for investment decision support~\cite{theil2019}.


\section{Experiments}


%Predicting Risk from Financial Reports with Regression 
There are similar papers in the literature~\cite{kogan2009}. We do not have many financial datasets for financial tasks.

We will compare our BERT based method with the following approaches:
\begin{itemize}
\item Baseline method will be volatility prediction based on GARCH similar to~\cite{kogan2009}.
\item SVM-based/Random Forest based volatility prediction as in
  paper~\cite{kogan2009}.
  \item SVM-based/Random Forest based return prediction as in
    paper. Previous papers has not focused on return prediction.
  \item Original BERT trained model.
   \item Elmo-based trained model.
\end{itemize}

One type of evaluation will based Mean squared error~(MSE) based.

Other evaluation type will be based on portfolio construction:
\begin{itemize}
\item Buckets on predicted returns. 
\item Stock portfolios based on Markovitz portfolios based on predicted returns and volatility and covariance between pairs.
\end{itemize}

We will focus on predicting the returns and volatilities for next
quarter, year half and half respectively. Portfolios will be balanced monthly.


Reuter's and Bloomberg dataset: Datasets are in Emre's email.
\url{https://github.com/philipperemy/financial-news-dataset}

Reuter's news dataset:
\url{https://github.com/duynht/financial-news-dataset}


\textbf{NLTK's corpus}
\url{https://www.kaggle.com/boldy717/reutersnltk#__sid=js0}


\textbf{Fed Meeting Notes}
\url{https://fraser.stlouisfed.org/title/federal-open-market-committee-meeting-minutes-transcripts-documents-677?browse=2020s}

FOMC Statements Scraper
https://github.com/souljourner/FOMC-Statements-Minutes-Scraper
Some cleaned transcripts
https://github.com/ali-wetrill/FOMCTranscriptAnalysis

We can predict volatility or whether volatility will go up or down of the following instruments:
\begin{itemize}
\item S\&P 500
\item 13-week Treasury Bills
\item 10-year Treasury Notes
\end{itemize}

Another source for datasets: \url{https://rstudio-pubs-static.s3.amazonaws.com/495650_c9c874694f164fb5948031801079157f.html#3_data}

We mainly focus on predicting the change of the Standard \& Poor’s 500 stock (S\&P 500) index,
obtaining indices and stock price data from Yahoo Finance. Standard \& Poor’s 500 is a stock market index based
on the market capitalizations of 500 large companies having common stock listed on the NYSE or NASDAQ.

\url{https://sraf.nd.edu/textual-analysis/resources/#LM%20Sentiment%20Word%20Lists}

Finally, the texts are stemmed using the Porter stemmer.

10K dataset together with volatilities
http://ifs.tuwien.ac.at/~admire/financialvolatility/


http://www.cs.cmu.edu/~ark/10K/data/
Metadata var

10K downloads: https://pypi.org/project/sec-edgar-downloader/. Filing
date is in each text file.

extract MDA and tokenize new files in Noah's website can be used to clean the dataset.
CIK Ticker Mapping: https://www.sec.gov/include/ticker.txt

Ticker price: Yahoo finance download https://towardsdatascience.com/downloading-historical-stock-prices-in-python-93f85f059c1f

This text regression problem has been discussed before.


\textbf{Corporate Reports 10-K \& 10-Q} The most important text data in finance and business communication is corporate report. In the United States,
the Securities Exchange Commission (SEC) mandates all publicly traded companies to file annual
reports, known as Form 10-K, and quarterly reports, known as Form 10-Q. This document provides a comprehensive overview of the company’s
business and financial condition. Laws and regulations prohibit companies from making materially
false or misleading statements in the 10-Ks. The
Form 10-Ks and 10-Qs are publicly available and can be accesses from
SEC website. We obtain 60,490 Form 10-Ks and 142,622
Form 10-Qs of Russell 3000 firms during 1994 and
2019 from SEC website. We only include sections that are textual components, such as Item 1 (Business) in 10-Ks, Item 1A (Risk Factors) in both 10-
Ks and 10-Qs and Item 7 (Managements Discussion and Analysis) in 10-Ks.


\bibliographystyle{plain}
\bibliography{returnpretrained}


\end{document}
